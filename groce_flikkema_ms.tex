\listfiles
\documentclass[manuscript, review, screen]{acmart}
%\setcitestyle{super,sort&compress}
\citestyle{acmauthoryear}
\usepackage{booktabs} % For formal tables

\usepackage[ruled]{algorithm2e} % For algorithms

% Metadata Information
%\acmJournal{TBD}
%\acmVolume{9}
%\acmNumber{4}
%\acmArticle{39}
%\acmYear{2010}
%\acmMonth{3}

%\acmBadgeL[http://ctuning.org/ae/ppopp2016.html]{ae-logo}
\acmBadgeR[http://ctuning.org/ae/ppopp2016.html]{ae-logo}

% Copyright
\setcopyright{acmcopyright}
%\setcopyright{acmlicensed}
%\setcopyright{rightsretained}
%\setcopyright{usgov}
%\setcopyright{usgovmixed}
%\setcopyright{cagov}
%\setcopyright{cagovmixed}

% DOI
\acmDOI{0000001.0000001}


% Document starts
\begin{document}
% Title portion
\title{Towards Testing and Testing Infrastructure to Support the CPS Design Process} 
% \titlenote{This is a titlenote}
% \subtitle{This is a subtitle}
% \subtitlenote{Subtitle note}
\author{Alex Groce}
\authornote{The corresponding author}
%\orcid{1234-5678-9012-3456}
\email{alex.groce@nau.edu}
\author{Paul G. Flikkema}
\email{paul.flikkema@nau.edu}
\affiliation{%
  \institution{Northern Arizona University}
  \department{School Informatics, Computing \& Cyber Systems}
  \city{Flagstaff}
  \state{AZ}
  \postcode{86001}
  \country{USA}
}

\begin{abstract}
%\footnote{}
\end{abstract}


%
% The code below should be generated by the tool at
% http://dl.acm.org/ccs.cfm
% Please copy and paste the code instead of the example below. 
%
\begin{CCSXML}
<ccs2012>
 <concept>
  <concept_id>10010520.10010553.10010562</concept_id>
  <concept_desc>Computer systems organization~Embedded systems</concept_desc>
  <concept_significance>500</concept_significance>
 </concept>
 <concept>
  <concept_id>10010520.10010575.10010755</concept_id>
  <concept_desc>Computer systems organization~Redundancy</concept_desc>
  <concept_significance>300</concept_significance>
 </concept>
 <concept>
  <concept_id>10010520.10010553.10010554</concept_id>
  <concept_desc>Computer systems organization~Robotics</concept_desc>
  <concept_significance>100</concept_significance>
 </concept>
 <concept>
  <concept_id>10003033.10003083.10003095</concept_id>
  <concept_desc>Networks~Network reliability</concept_desc>
  <concept_significance>100</concept_significance>
 </concept>
</ccs2012>  
\end{CCSXML}

%\ccsdesc[500]{Computer systems organization~Embedded systems}
%\ccsdesc[300]{Computer systems organization~Redundancy}
%\ccsdesc{Computer systems organization~Robotics}
%\ccsdesc[100]{Networks~Network reliability}

%
% End generated code
%


\keywords{TBD}


%\thanks{}


\maketitle

\section{Introduction}

Cyber-physical systems exhibit a number of characteristics that challenge current design paradigms.  We distinguish here between three types of characteristics. Key characteristics inherent to deployed systems include the need for reactive or real time processing; networks that inject stochastic delays and loss of reliability in communication; nodes with strongly heterogeneous processing platforms; and brittle interfaces. Growing demand for distributed intelligence in CPS will increase the complexity of CPS software. These characteristics in turn influence CPS design and design processes through layering and the refinement of abstractions. To manage design complexity, CPS design employs layering in multiple domains, e.g., computation, networking, and the modeling of the embedding physical system and the system's sensors and actuators. However, current layered do not capture non-functional system properties essential to CPS, e.g., timing and energy use, that emerge via testing.  To manage design process complexity, iterative development is commonplace: while the long-term trend is refinement of abstract models, engineers would like to shuffle between implementation-level models and more abstract models to gather new data, gain knowledge and insight, and optimize system performance. 

The integration of testing into the design process will be central to the success of CPS in complexity-demanding applications, but the question is how to do this. We propose an architectural model of testing for CPS that enables combining tests in two dimensions: the domain, and the level of abstraction (layer) in that domain.  The key point is to construct test architectures  that enable the ability to employ constituent CPS models for different domains at diverse levels of abstraction in a plug-n-play manner.

As just one case, it would be helpful to run the low-level tests in one domain in the context of other domains expressed using high-level models. For example, a unit-level test of implementation-level code running on a target sensor-actuator node might be connected with a meteorological model running in the cloud that drives an emulated sensor, and, through a simple packet-layer communication model, to a high-level control system model running in real-time on an engineer's workstation that in turn drives an actuator via commands sent through the communication model. As another example, existing production-level, server-based control code may need to be integrated with a new actuating subsystem. Here, functional tests could be performed using a sequence of models (of increasing refinement) prior to integration of the target actuating hardware.

MORE HERE.

\section{Concluding Remarks}

TBD.

%\end{document}  % This is where a 'short' article might terminate



%\appendix
%%Appendix A
%\section{Headings in Appendices}
%The rules about hierarchical headings discussed above for
%the body of the article are different in the appendices.
%In the \textbf{appendix} environment, the command
%\textbf{section} is used to
%indicate the start of each Appendix, with alphabetic order
%designation (i.e., the first is A, the second B, etc.) and
%a title (if you include one).  So, if you need
%hierarchical structure
%\textit{within} an Appendix, start with \textbf{subsection} as the
%highest level. Here is an outline of the body of this
%document in Appendix-appropriate form:
%\subsection{Introduction}
%\subsection{The Body of the Paper}
%\subsubsection{Type Changes and  Special Characters}
%\subsubsection{Math Equations}
%\paragraph{Inline (In-text) Equations}
%\paragraph{Display Equations}
%\subsubsection{Citations}
%\subsubsection{Tables}
%\subsubsection{Figures}
%\subsubsection{Theorem-like Constructs}
%\subsubsection*{A Caveat for the \TeX\ Expert}
%\subsection{Conclusions}
%\subsection{References}
%Generated by bibtex from your \texttt{.bib} file.  Run latex,
%then bibtex, then latex twice (to resolve references)
%to create the \texttt{.bbl} file.  Insert that \texttt{.bbl}
%file into the \texttt{.tex} source file and comment out
%the command \texttt{{\char'134}thebibliography}.
%% This next section command marks the start of
%% Appendix B, and does not continue the present hierarchy
%\section{More Help for the Hardy}
%
%Of course, reading the source code is always useful.  The file
%\path{acmart.pdf} contains both the user guide and the commented
%code.
%
%\begin{acks}
%  The authors would like to thank Dr. Yuhua Li for providing the
%  matlab code of  the \textit{BEPS} method. 
%
%  The authors would also like to thank the anonymous referees for
%  their valuable comments and helpful suggestions. The work is
%  supported by the \grantsponsor{GS501100001809}{National Natural
%    Science Foundation of
%    China}{http://dx.doi.org/10.13039/501100001809} under Grant
%  No.:~\grantnum{GS501100001809}{61273304}
%  and~\grantnum[http://www.nnsf.cn/youngscientsts]{GS501100001809}{Young
%    Scientsts' Support Program}.
%
%\end{acks}



\end{document}
